Recently, the importance of network operation and management has been emphasized due to the occurrence of various services and applications.
Especially due to the rapid growth of IoT, various related applications and services are being provided through the network.
Therefore, the network operator needs differentiated services according to applications provided through the network.
In particular, video and voice services require fast network services.
On the other hand, services such as text can provide enough services without having a fast transmission speed.
In addition, Peer-to-Peer (P2P) services such as BitTorrent take up a significant portion of the world's Internet traffic, which has a significant impact on overall network speed.
Thus, according to each service, the network operator tries to provide smooth Quality of Service (QoS) by placing different priority.

Research on how to classify network traffic in order to provide smooth services according to application programs is actively being conducted.
There are various ways to classify network traffic, such as rule-based, port-based, flow-correlation-based, payload-based and using deep-learning.
In general, rule-based and port-based network traffic classification methods are widely used.
However, rule-based and port-based network traffic classification methods are classified by well-known application port numbers or specific rules but are not well classified for unknown applications because they do not know the port number or specific rules.
Also, a network operator has the inconvenience of manually adding a rule or port number in order to provide a new network service.

The payload-based network classification method classifies only the pure application layer payload information excluding the packet header information of the entire network traffic.
There is a problem with locality dependency when packet classification is performed by adding header information.
Also, if the header information is changed, classification will not be performed properly.
But, the payload-based network classification method solves the locality dependency problem because it uses only the payload data, and it is not affected even if the header information is changed.
Additionally, current payload-based methods provide the best classification accuracy.
However, there is a practical problem due to the difficulty of accessing the payload of the raw packet and the encryption due to the user privacy policy.

Recently, researches on the field of deep learning have progressed actively and have become applicable to various fields.
So, there is much research on network traffic classification using deep learning.
In the network traffic classification study using the existing deep learning model, classification is performed using the header information of the packet as a feature.
Therefore, there is a limitation because the classification is performed within a local network.
In a real network that is outside the local limits, it is difficult to classify through the previously learned model.

In this paper, the datasets of the deep learning model create training datasets from the network traffic through the self-developed data preprocessing.
Through preprocessing, one packet in network traffic is imaged and generated as training data.
The imaged packets are gathered for each application and are made up of training datasets of packet unit or flow unit.
Convolution Neural Network (CNN), Residual Network (ResNet), Recurrent Neural Network (RNN), Long Short-Term Memory (LSTM), and CNN + RNN are learned to classify network traffic using the generated packet unit and flow unit training datasets.
CNN and ResNet models are suitable for classification of image data, they are used for learning imaged datasets of packet unit.
The RNN, LSTM, and CNN + RNN learning models are suitable for learning sequential datasets.
Therefore, They are used to learn flow unit datasets that contain time sequential of network traffic.

To enhance the performance of network traffic classification, we added a model tuning procedure to find optimal hyper-parameters of each deep learning model.
Then, we compare the packet unit and the flow unit datasets by using five different deep learning models.

The remainder of this paper is organized as follows.
Section II describes related work and introduces motivation for our work.
Section III describes the deep learning models architecture for network traffic classification.
Section IV presents the model tuning method for optimal hyper-parameter.
Section V shows experiment results and analysis.
Section VI provides concluding remarks.