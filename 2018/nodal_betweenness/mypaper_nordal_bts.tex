%%
%% Copyright 2007-2018 Elsevier Ltd
%% 
%% This file is part of the 'Elsarticle Bundle'.
%% ---------------------------------------------
%% 
%% It may be distributed under the conditions of the LaTeX P{sec:betweenness}
%% License, either version 1.2 of this license or (at your option) any
%% later version.  The latest version of this license is in
%%    http://www.latex-project.org/lppl.txt
%% and version 1.2 or later is part of all distributions of LaTeX
%% version 1999/12/01 or later.
%% 
%% The list of all files belonging to the 'Elsarticle Bundle' is
%% given in the file `manifest.txt'.
%% 
%% Template article for Elsevier's document class `elsarticle'
%% with harvard style bibliographic references

\documentclass[preprint,12pt,authoryear]{elsarticle}

%% Use the option review to obtain double line spacing
%% \documentclass[authoryear,preprint,review,12pt]{elsarticle}

%% Use the options 1p,twocolumn; 3p; 3p,twocolumn; 5p; or 5p,twocolumn
%% for a journal layout:
%% \documentclass[final,1p,times,authoryear]{elsarticle}
%% \documentclass[final,1p,times,twocolumn,authoryear]{elsarticle}
%% \documentclass[final,3p,times,authoryear]{elsarticle}
%% \documentclass[final,3p,times,twocolumn,authoryear]{elsarticle}
%% \documentclass[final,5p,times,authoryear]{elsarticle}
%% \documentclass[final,5p,times,twocolumn,authoryear]{elsarticle}

%% For including figures, graphicx.sty has been loaded in
%% elsarticle.cls. If you prefer to use the old commands
%% please give \usepackage{epsfig}

%% The amssymb package provides various useful mathematical symbols
\usepackage{amssymb}

\usepackage{float}%exact location
\usepackage{graphicx}
\usepackage{caption}
\usepackage{subcaption}

\usepackage{hhline}
\usepackage{multirow}
\usepackage[utf8]{inputenc}
\usepackage[T1]{fontenc}
\usepackage{lmodern}
\usepackage{epsfig}
\usepackage{amsthm}
\usepackage{array}
\usepackage{algorithmicx}
\usepackage[ruled]{algorithm}
\usepackage{algpseudocode, caption}
 
\usepackage[cmex10]{amsmath}
%% The amsthm package provides extended theorem environments
%% \usepackage{amsthm}

%% The lineno packages adds line numbers. Start line numbering with
%% \begin{linenumbers}, end it with \end{linenumbers}. Or switch it on
%% for the whole article with \linenumbers.
%% \usepackage{lineno}

\mathchardef\mhyphen="2D

\newcommand{\D}[3]{\delta_{{#1}{#2}}({#3})}
\newcommand{\V}[2]{V_{#2}({#1})}
\newcommand{\LV}[2]{V_{\le #2}({#1})}
\newcommand{\E}[2]{E_{#2}({#1})}
\newcommand{\LE}[2]{E_{\le #2}({#1})}
\newcommand{\EH}[3]{E_{{\le #2}:{\le #3}}({#1})}

\newcommand{\EN}[1]{\mathcal{E}_{{#1}}}
\newcommand{\ENN}[2]{\mathcal{E}_{{#1}}^{{#2}}}
\newcommand{\FNN}[2]{\mathcal{F}_{{#1}}^{{#2}}}
\newcommand{\XN}[1]{\mathcal{X}_{{#1}}}
\newcommand{\CC}[1]{\mathcal{C}_{{#1}}}

\newcommand{\B}[1]{B({#1})}
\newcommand{\BE}[2]{B_{{#1}}^{\mathcal{E}}({#2})}
\newcommand{\BF}[2]{B_{{#1}}^{\mathcal{F}}({#2})}
\newcommand{\BX}[1]{B^{\mathcal{X}}({#1})}

\newcommand{\NX}{n_{x}}
\newcommand{\MX}{m_{x}}
\newcommand{\NE}{n_{e}}
\newcommand{\ME}{m_{e}}
\newcommand{\NC}{n_{c}}
\newcommand{\MC}{m_{c}}


\theoremstyle{definition}
\newtheorem{definition}{Definition}[section]

\newtheorem{theorem}{Theorem}[section]
\newtheorem{example}[theorem]{Example}

\journal{Social Networks}

\begin{document}

\begin{frontmatter}

%% Title, authors and addresses

%% use the tnoteref command within \title for footnotes;
%% use the tnotetext command for theassociated footnote;
%% use the fnref command within \author or \address for footnotes;
%% use the fntext command for theassociated footnote;
%% use the corref command within \author for corresponding author footnotes;
%% use the cortext command for theassociated footnote;
%% use the ead command for the email address,
%% and the form \ead[url] for the home page:
%% \title{Title\tnoteref{label1}}
%% \tnotetext[label1]{}
%% \author{Name\corref{cor1}\fnref{label2}}
%% \ead{email address}
%% \ead[url]{home page}
%% \fntext[label2]{}
%% \cortext[cor1]{}
%% \address{Address\fnref{label3}}
%% \fntext[label3]{}

\title{A Fast Algorithm for Nodal Betweenness Centraility}

%% use optional labels to link authors explicitly to addresses:
%% \author[label1,label2]{}
%% \address[label1]{}
%% \address[label2]{}

\author{}

\address{}

\begin{abstract}
In this section, we define terms that represent two types of egocentric networks, egocentric networks and multi-layered egocentric networks (Section~\ref{x-ego_definition}), as well as the betweenness of a vertex in its ego and x-egocentric networks (Section~\ref{sec:betweenness}).
We then present several properties of the multi-layered egocentric networks (Section~\ref{x-go_properties}).
Our betweenness computation algorithm in Section~\ref{computation} takes advantage of these properties.


\end{abstract}

\begin{keyword}
Egocentric \sep Friendship Networks \sep Betweenness Centrality
%% keywords here, in the form: keyword \sep keyword

%% PACS codes here, in the form: \PACS code \sep code

%% MSC codes here, in the form: \MSC code \sep code
%% or \MSC[2008] code \sep code (2000 is the default)

\end{keyword}

\end{frontmatter}

%% \linenumbers

%% main text
\section{Introduction}\label{sec:introduction}

In social network analysis, the betweenness of an actor indicates the extent to which that actor is between all other actors within the network (\cite{centrality}).

Calculating the betweenness of each node, however, requires finding all of the shortest paths between every pair of nodes in the given network.
Since carrying out this task in a large wireless network will incur prohibitively expensive computational costs, techniques for estimating betweenness have been developed~\cite{SIMBET,egocentric,everett,ICCN:lbcdna,Pant13:Local}.
In these techniques, each node identifies its \emph{ego network}, a logical network consisting of that node, its 1-hop neighbors, and all links between these nodes.
Then, each node calculates, as an estimate of its betweenness in the entire network, its betweenness only in its ego network, thereby saving both network and computational resources.


\section{Multi-order Egocentric Friendship Networks and Their Betweenness}\label{sec:multi-order-friendship-network}

\begin{table}[t]
\captionsetup[subfigure]{aboveskip=-2pt, belowskip=-1pt}
\center
\caption{Summary of notation}\label{table:symbols}
\small
    \begin{tabular}{| p{1.2cm} | p{10cm} |}
        \hline
        Symbol & Description \\
        \hline
        \hline
        $V/\{v\}$ & set of all vertices except $v$ \\        
        $\V{v}{i}$ & set of $i$-hop neighbors of vertex $v$ ($\V{v}{0} = \{ v \}$)\\
        $\LV{v}{i}$ & set of vertices that are at most $i$ hops away from vertex $v$ (i.e., $\LV{v}{i} = \cup_{k = 0}^{i} \V{v}{k}) $\\
        $\E{v}{i}$ & set of edges connecting two vertices in $\V{v}{i}$ ($\E{v}{0}$=$\emptyset$)\\
        $\LE{v}{i}$ & set of edges connecting two vertices in $\LV{v}{i}$\\
        \hline
    \end{tabular}
\end{table}

\section{Betweenness in Multi-order Friendship Networks}\label{sec:betweenness}

\alglanguage{pseudocode}
\begin{algorithm}[t]
\label{alg:main}
\begin{algorithmic}[1]
\State $\mathbf{Input}$: a graph $G(V,E)$ and a vertex $v (\in V)$
\State $\mathbf{Output}$: $betweenness\ of\ {v}$
\State create an $2$-dimensional array $D[1,2,...,|V|][1,2,...,|V|]$ 
\State $sum \gets 0$
\For{$p$ : $1$ to $u$}
	\For{$q$ : $p+1$ to $u$} 
		\State $D[p][q] \gets dependency1(p, q, N)$
		\State $sum \gets sum + D[p][q]$
	\EndFor
	\For{$q$ : $u+1$ to $u+w$} 
		\State $D[p][q] \gets dependency2(p, q, N, D)$
		\State $sum \gets sum + D[p][q]$
	\EndFor
\EndFor
\For{$p$ : $u+1$ to $u+w$} 
	\For{$q$ : $p+1$ to $u+w$}
		\State $D[p][q] \gets dependency2(p, q, N, D)$
		\State $sum \gets sum + D[p][q]$
	\EndFor
\EndFor
\State \Return{$\frac{2 \cdot sum}{(u+w)(u+w-1)}$} 
\caption{$nodal\_betweenness(G(V,E), v)$}\label{alg:main}

\end{algorithmic}
\end{algorithm}
\alglanguage{pseudocode}
\begin{algorithm}[t]
\begin{algorithmic}[1]
\State $\mathbf{Input}$: $p$, $q$, $N[0,1,...,u+w]$
\If{$q \in N[p]$}
	\State \Return{$0$} \Comment{by Theorem 1}
\Else
	\State \Return{$1/{\mid N[p] \cap N[q] \mid}$} \Comment{by Theorem 2}
\EndIf
\caption{$dependency1(p, q, N)$}
\label{alg:dependency1}
\end{algorithmic}
\end{algorithm}
\alglanguage{pseudocode}
\begin{algorithm}[t]
\begin{algorithmic}[1]
\State $\mathbf{Input}$: $p$, $q$, $N[0,1,...,u+w]$, $D[1,2,...,u+w][1,2,...,u+w]$
\State $C \gets []$\;
\For{each $r \in N[q]$}
	\If {$p < r$}
		\State $\tau = D[p][r]$
	\Else 
		\State $\tau = D[r][p]$
	\EndIf
	\If {$\tau==0$}
		\State \Return{0} \Comment{by Theorem 3}
	\Else 
		\State append $\tau$ to $C$
	\EndIf
\EndFor
\State \Return{$\bar{H}(C)$}    \Comment{by Theorem 4}
\caption{$dependency2(p, q, N, D)$}
\label{alg:dependency2}
\end{algorithmic}
\end{algorithm}

\section{Evaluation}\label{evaluation}

\section{EEGO}\label{eego}

\section{Computation}\label{computation}


%% The Appendices part is started with the command \appendix;
%% appendix sections are then done as normal sections
%% \appendix

%% \section{}
%% \label{}

%% If you have bibdatabase file and want bibtex to generate the
%% bibitems, please use
%%
\bibliographystyle{elsarticle-num-names}
\bibliography{../../LinkPaper}

%% else use the following coding to input the bibitems directly in the
%% TeX file.

%\begin{thebibliography}{00}

%% \bibitem[Author(year)]{label}
%% Text of bibliographic item

%\bibitem[ ()]{}

%\end{thebibliography}
\end{document}

\endinput
%%
%% End of file `elsarticle-template-harv.tex'.