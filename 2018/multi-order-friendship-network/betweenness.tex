In this section, we propose a new algorithm to calculate the betweenness centrality in the multi-oder friendship
networks.
\subsection{Ego and X-Ego Betweenness}\label{betweenness-sub}
\begin{center}
\begin{table*}[t]
 \caption{Comparison of betweenness, ego betweenness, and x-ego betweenness for the graph shown in Fig. \ref{eego} (the Pearson correlation is 0.63 between $\B{v}$ and $\BE{v}$ and 0.90 between $\B{v}$ and $\BX{v}$, and the Spearman correlation is 0.79 between $\B{v}$ and $\BE{v}$ and 0.93 between $\B{v}$ and $\BX{v}$)}\label{comparison}
 \resizebox{17.4cm}{!} {
 \begin{tabular}{|c|c|c|c|c|c|c|c|c|c|c|c|c|c|c|}
 \hline
\multicolumn{2}{|c|}{Nodes} & v & a & b & c & d & e & f & g & h & i & j & k & l \\
 \hline
\multirow{2}{*}{$\B{v}$} & Value & 0.405 & 0.383 & 0.124 & 0.131 & 0.286 & 0.000 & 0.318 & 0.049 & 0.030 & 0.167 & 0.000 & 0.000 & 0.000\\
\hhline{~--------------}
& Rank & 1 & 2 & 7 & 6 & 4 & 10 & 3 & 8 & 9 & 5 & 10 & 10 & 10\\
\hhline{---------------}
\multirow{2}{*}{$\BE{v}$} & Value & 0.667 & 0.750 & 0.167 & 0.583 & 0.333 & 0.000 & 0.833 & 1.000 & 0.167 & 0.667 & 0.000 & 0.000 & 0.000\\
\hhline{~--------------}
& Rank & 4 & 3 & 8 & 6 & 7 & 10 & 2 & 1 & 8 & 4 & 10 & 10 & 10\\
\hhline{---------------}
\multirow{2}{*}{$\BX{v}$} & Value & 0.383 & 0.500 & 0.125 & 0.269 & 0.339 & 0.000 & 0.524 & 0.214 & 0.133 & 0.400 & 0.000 & 0.000 & 0.000\\
\hhline{~--------------}
& Rank & 4 & 2 & 9 & 6 & 5 & 10 & 1 & 7 & 8 & 3 & 10 & 10 & 10\\
\hhline{---------------}
 \hline
 \end{tabular}
}
\end{table*}
\end{center}
% In much literature, the betweenness $C_{B}(i)$ for a node $v$ has been formally defined as follows:\begin{equation}\label{bet}
% C_{B}(v) = \frac{\sum_{s \neq v \neq t \in V, s<t}\frac{\sigma_{st}(v)}{\sigma_{st}}}{(n-1)(n-2)/2}.
% \end{equation} where $n$ is the total number of nodes, $\sigma_{st}$ is the number of shortest paths from $s$ to $t$, and $\sigma_{st}(i)$ is the number of those shortest paths that include the node $v$. In an undirected network, $\sigma_{st}=\sigma_{ts}$ and $\sigma_{st}(i)=\sigma_{ts}(i)$. The denominator denotes the total number of pairs of nodes except $v$ in an undirected network and normalizes $C_{B}(v)$ to a value between 0 and 1.
In the literature~\cite{centrality, everett, Brandes01afaster}, given a graph $G(V, E)$, the betweenness $\B{v}$ of a vertex $v$ is defined as:
\begin{equation}\label{bet}
\B{v} = \frac{\sum_{s \neq v \neq t \in V}\frac{\sigma_{st}(v)}{\sigma_{st}}}{(|V|-1)(|V|-2)}
\end{equation} where $\sigma_{st}$ is the number of shortest paths from vertex $s$ to vertex $t$ and $\sigma_{st}(v)$ is the number of those shortest paths that pass through vertex $v$. 
% \B{v} = \frac{2 \cdot \sum_{s \neq v \neq t \in V,\ s < t}\frac{\sigma_{st}(v)}{\sigma_{st}}}{(|V|-1)(|V|-2)}.
% \end{equation} where $\sigma_{st}$ is the number of shortest paths from vertex $s$ to vertex $t$ and $\sigma_{st}(v)$ is the number of those shortest paths that pass through vertex $v$. 
In the above definition, the denominator represents the total number of pairs of vertices except $v$.
It normalizes $\B{v}$ to a value between 0 and 1.
Given an undirected graph, $\sigma_{st}=\sigma_{ts}$ and $\sigma_{st}(v)=\sigma_{ts}(v)$ for all vertices $s$, $t$, and $v$.
Therefore, it is sufficient to find either $\sigma_{st}$ or $\sigma_{ts}$ and either $\sigma_{st}(v)$ or $\sigma_{ts}(v)$.
% The globally computed betweenness based on (\ref{bet}) requires much information about the whole network topology and causes large computational and message load overheads. Since a DTN node may be energy-constrained and there is usually no explicit centralized network node in a DTN, such computation is not reasonable. Thus, getting local information from encounter nodes and generating ego or x-ego networks in a distributed way are practically feasible.
% 
% A node $v$'s \emph{ego betweenness} $C^{\mathcal{E}}_{B}(v)$ on its ego network $\EN{v}$ is locally computed as follows:\begin{equation}\label{ebet}
% C^{\mathcal{E}}_{B}(v) = \frac{\sum_{s \neq v \neq t \in V_{1}^{v}, s<t}\frac{\sigma_{st}(v)}{\sigma_{st}}}{(|V_{1}^{v}|-1)(|V_{1}^{v}|-2)/2}. 
% \end{equation} and the \emph{x-ego betweenness} $C^{\mathcal{X}}_{B}(v)$ on its x-ego network $\XN{v}$ is also locally obtained as follows: \begin{equation}\label{eebet}
% C^{\mathcal{X}}_{B}(v) = \frac{\sum_{s \neq v \neq t \in V_{2}^{v}, s<t}\frac{\sigma_{st}(v)}{\sigma_{st}}}{(|V_{2}^{v}|-1)(|V_{2}^{v}|-2)/2}.
% \end{equation} Although the x-ego betweenness requires more computational time than the ego betweenness, it has the similar benefit of simplicity with the ego network in terms of information collection and network configuration. On the other hand, the x-ego betweenness includes more local information than the ego network, so that it will be more accurate than the ego betweenness. 
In a large wireless network, obtaining the betweenness of each node (i.e., the betweenness of the vertex representing that node) is costly since  all nodes that have limited memory and energy must exchange and consume a substantial number of messages to identify the shortest paths between them.
On the other hand, each node can obtain its ego and x-ego networks with much lower overhead since each node  needs to broadcast information about its 1-hop neighbors (Section~\ref{x-ego_definition}). 
In this paper, we consider the situations where each node computes its betweenness using either its ego network or x-ego network and then uses the result as an estimate of its true betweenness in the entire network.
We refer to the betweenness of $v$ computed from the ego network and x-ego network of $v$ as the {\em ego betweenness} and {\em x-ego betweenness} of $v$ (denoted $\BE{v}$ and $\BX{v}$), respectively.
% To compare the accuracy of $\BE{v}$ or $\BX{v}$, we first use the Spearman's rank correlation which measures the {\em monotonic} relationship between the ranked variables of $\BX{v}$ and $\B{v}$ or $\BE{v}$ and $\B{v}$.  The correlation value can vary between -1.0 and +1.0. The closer value is to +1.0, the stronger the association between the values. For the sake of completeness, we also measure the well-known Pearson correlation. The Pearson correlation coefficient assesses a {\em linear} relationship between the two variables and the calculation is based not on the ranked variables, but on the actual betweenness values.
% For the networks shown in Fig. \ref{}, Table \ref{comparison} shows the comparison of globally computed, ego, and x-ego betweenness. Although the actual betweenness values are different with each other, we can see that 1) the rank order of the three betweenness values are very similar to each other, and 2) the rank order of x-ego betweenness is more similar to the one of globally computed betweenness than the one of ego betweenness. We will verify the details with real mobility trace data in Section IV.
Table~\ref{comparison} shows, for every vertex $v$ in Fig.~\ref{eego}, the betweenness ($\B{v}$), ego betweenness ($\BE{v}$), and x-ego betweenness ($\BX{v}$).
In this table, for most of the vertices, x-ego betweenness is closer to betweenness than ego-betweenness mainly because it is derived from a larger number of vertices and edges.
For this reason, the correlation coefficient (also known as the Pearson correlation coefficient) of x-ego betweenness and betweenness (0.9) is higher than that of ego-betweenness and betweenness (0.63).
The advantage of x-ego betweenness over ego betweenness can also be observed in terms of Spearman's rank correlation, which indicates, given two series $\mathcal{X}=(X_1, X_2, \cdots, X_N)$ and $\mathcal{Y}=(Y_1, Y_2, \cdots, Y_N)$, the correlation between $(r_{\mathcal{X}}(X_1), r_{\mathcal{X}}(X_2)$ $, \cdots, r_{\mathcal{X}}(X_N))$ and $(r_{\mathcal{Y}}(Y_1), r_{\mathcal{Y}}(Y_2), \cdots, r_{\mathcal{Y}}(Y_N))$, where $r_{\mathcal{X}}(X_i)$ and $r_{\mathcal{Y}}(Y_i)$ represent the rank of $X_i$ in $\mathcal{X}$ and that of $Y_i$ in $\mathcal{Y}$, respectively.
In Table~\ref{comparison}, Spearman's correlation is 0.93 between x-ego betweenness and betweenness and 0.79 between ego betweenness and betweenness.
In Section~\ref{evaluation}, we further demonstrate the benefit of x-ego betweenness using wireless trace data.
\subsection{Properties of X-Ego Networks}\label{x-go_properties}
In this section, we present four properties of x-ego networks. 
These properties enable efficient x-ego betweenness computation (Section~\ref{computation}). 
As in Brandes' work~\cite{Brandes01afaster}, we denote the \emph{dependency} of vertices $s$ and $t$ on vertex $v$ as $\D{s}{t}{v} = \frac{\sigma_{st}(v)}{\sigma_{st}}$.
Then, the betweenness of $v$ (Equation (\ref{bet})) can be computed by adding the dependency values for all pairs of vertices excluding $v$.
To quickly calculate such dependency values in x-ego networks, we have identified the properties explained below. 
% By Equation (\ref{bet}), we can know that the betweenness of $v$ can be obtained by summing the dependencies for all pairs of vertices, except for $v$, in $\XN{v}$.
% % The exhaustive search of such all shortest paths is computationally expensive. 
% We identify the following properties, {\em Theorem \ref{theorem1}, \ref{theorem2}, \ref{theorem3}}, and  {\em \ref{theorem4}}, to obtain such dependencies with low computational overhead.
% % Therefore, we need to devise {\em skip conditions} or (\ldots) to reduce the computational burden.  
\begin{theorem}
\label{theorem1} 
Assume a vertex $v$, its x-ego network $\XN{v}$, and its two different 1-hop neighbors $s$ and $t$ (i.e., $s, t \in \V{v}{1}$ and $s \ne t$).
Then, $\D{s}{t}{v} = 0$ if there is an edge between $s$ and $t$ (i.e., $\{s, t\} \in \E{v}{1}$).
\begin{proof}
Since $\{s, t\} \in \E{v}{1}$, $d(s, t) = 1$.
Furthermore, $s, t \in \V{v}{1}$, meaning that $d(s, v) + d(v, t) = 1 + 1 = 2 > d(s, t) = 1$ (i.e., no shortest path from $s$ to $t$ passes through $v$).
Therefore, $\D{s}{t}{v} = \sigma_{st}(v)/\sigma_{st} = 0/\sigma_{st} = 0$.\hfill\qed
\end{proof}
\end{theorem}
\begin{example}
In Fig.~\ref{eego2}(b), $\D{a}{b}{v} = 0$ since $a, b \in \V{v}{1}$ and there is an edge between $a$ and $b$.
\end{example}
% Theorem \ref{theorem1} will take on a key role as a computation skipping rule in the algorithm to compute the x-ego betweenness, $\BX{v}$. In particular, it will reduce the computational overload substantially when a vertex $v$'s x-ego network exhibits a strong community structure (i.e., nodes in ego networks tend to interact with each other). Since friends of a friend are likely to be friends as well, we can expect that this transitivity should be frequently observed in x-ego networks derived by a DTN. 
% 
% In case that the skipping rule presented by Theorem \ref{theorem1} cannot be applied for a pair of $v$'s two 1-hop neighbors, the dependency of them is computed by using the following Theorem \ref{theorem2}. 
Theorem \ref{theorem1} allows us to quickly compute dependencies particularly when x-ego networks exhibit a strong community structure (i.e., 1-hop neighbors of a node tend to have direct social links between them).
In Section~\ref{evaluation}, we show this benefit using wireless trace data.
When Theorem \ref{theorem1} cannot be applied to $v$'s 1-hop neighbors $s$ and $t$ (i.e., there is no edge between $s$ and $t$), the dependency of $s$ and $t$ on $v$ can be computed using the following theorem.
\begin{theorem}
\label{theorem2} 
Assume a vertex $v$, its x-ego network $\XN{v}$, and its two different 1-hop neighbors $s$ and $t$ (i.e., $s, t \in \V{v}{1}$ and $s \ne t$).
Then, $\D{s}{t}{v} = {1}/{|\V{s}{1} \cap \V{t}{1}|}$ if there is no edge between $s$ and $t$ (i.e., $\{s, t\} \notin \E{v}{1}$).
\begin{proof}
Since $\{s, t\} \notin \E{v}{1}$, $d(s, t) > 1$.
In this case, $d(s, t) = 2$ due to the path $s - v - t$.
Furthermore, (i) the number of shortest paths from $s$ to $t$ (i.e., paths whose length is 2) can be expressed as $\sigma_{st} = |\V{s}{1} \cap \V{t}{1}|$.
%  indicates the number of shortest paths starting from $v_s \in N_0$ and terminating at $v_t \in N_0$.
On the other hand, (ii) the path $s-v-t$ is the only shortest path from $s$ to $t$ that passes through $v$ (i.e., $\sigma_{st}(v) = 1$) since the length of that path is 2, which must be smaller than the length of any other path from $s$ to $t$ that passes through $v$.
By (i) and (ii), $\D{s}{t}{v} = \sigma_{st}(v)/\sigma_{st} = {1}/{|\V{s}{1} \cap \V{t}{1}|}$.\hfill\qed
\end{proof}
\end{theorem}
% \begin{equation}
% \D{s}{t}{v} =
%   \begin{cases}
%    0       & \text{if $(s, t) \in E$} \\
%    {\displaystyle \frac {1}{|\V{s}{1} \cap \V{t}{1}|}} & \text{ otherwise.} \\
%   \end{cases}
% \end{equation}
\begin{example}
\label{example2} 
In Fig.~\ref{eego2}(b), vertices $a$ and $c$ are 1-hop neighbors of $v$ and there is no edge between $a$ and $c$.
Furthermore, $\V{a}{1} = \lbrace b, e, f, g, v\rbrace$ and $\V{c}{1} = \lbrace d, g, h, v\rbrace$.
Therefore, $\D{a}{c}{v}=1/|\V{a}{1} \cap \V{c}{1}|=1/|\{g, v \}|=1/2$.
% $\LV{a}{1} = \lbrace v, b, e, f, g\rbrace$, $\LV{c}{1} = \lbrace v, d, g, h\rbrace$ and $\LV{d}{1} = \lbrace v, c, h, i\rbrace$. So, node $v$ can get the non-zero $\D{a}{c}{v}=1/2=0.5$ and $\D{a}{d}{v}=1/1=1$ because $v$ knows that $a$ does not share any edge with $c$ and $d$, $|\LV{a}{1} \cap \LV{c}{1}| = |\lbrace v, g\rbrace|=2$, and $|\LV{a}{1} \cap \LV{d}{1}| = |\lbrace v\rbrace$| = 1. 
\end{example}
Just like Theorem~\ref{theorem1}, the following theorem quickly computes the dependency of two vertices on a vertex $v$.
While the former applies to a pair of 1-hop neighbors of $v$, the latter applies to a pair of a 1-hop or 2-hop neighbor and a 2 hop-neighbor of $v$.
% The following Theorem \ref{theorem3} presents another skipping rule to reduce the computation overload of the dependency for an 1 or 2-hop neighbor and a 2-hop neighbor of $v$.
\begin{theorem}
\label{theorem3} 
Assume a vertex $v$, its x-ego network $\XN{v}$, a vertex $s \in \LV{v}{2}$, and another vertex $t \in \V{v}{2}$ such that $s \ne t$.
Then, $\D{s}{t}{v} = 0$ if $\D{s}{n}{v} = 0$ for a 1-hop neighbor vertex $n$ of vertex $t$ (i.e., for $n \in \V{t}{1}$).
\begin{proof}
Since $\D{s}{n}{v} = 0$ (i.e., no shortest path from $s$ to $n$ passes through $v$), (i) $d(s, n) < d(s, v) +d(v, n)$.
Then, (ii) $d(s, t) \le d(s, n) + d(n, t) < d(s, v) + d(v, n) + d(n, t)$ by (i).
Furthermore, $n \in \V{t}{1}$ and $t \in \V{v}{2}$, meaning that vertex $n$ is either a 1-hop or a 3-hop neighbor of $v$.
In $\XN{v}$, however, any vertex including $n$ is at most 2 hops away from $v$.
For this reason, $n$ is a 1-hop neighbor of $v$.
Since $d(v, n)$=$d(n, t)$=$1$ and $d(v, t)=2$, (iii) $d(v, n) + d(n, t) = d(v, t)$. 
By (ii) and (iii), $d(s, t) < d(s, v) +d(v, t)$ (i.e., no shortest path from $s$ to $t$ passes through $v$).
Therefore, $\D{s}{t}{v}  = \sigma_{st}(v)/\sigma_{st} = 0/\sigma_{st} = 0$.\hfill\qed
\end{proof}
\end{theorem}
\begin{example}
In Fig. \ref{eego2}(b), $b \in \LV{v}{2}$ and $g \in \V{v}{2}$. 
Furthermore, for a 1-hop neighbor $a$ of $g$, $\D{b}{a}{v}=0$ (Theorem \ref{theorem1}).
Therefore, by Theorem \ref{theorem3}, $\D{b}{g}{v}=0$.
\end{example}
Given vertices $s \in \LV{v}{2}$ and $t \in \V{v}{2}$ such that $s \ne t$, Theorem~\ref{theorem3} cannot be applied if $\D{s}{n}{v} > 0$ for every 1-hop neighbor $n$ of $t$.
In this case, the dependency of $s$ and $t$ on $v$ can be obtained using the following theorem.
% Lastly, the following Theorem \ref{theorem4} presents that the harmonic mean can be used to compute the dependency for an 1 or 2-hop neighbor and a 2-hop neighbor of $v$, if they do not satisfy the condition of Theorem \ref{theorem3}. 
\begin{theorem}
\label{theorem4} 
Assume a vertex $v$, its x-ego network $\XN{v}$, a vertex $s \in \LV{v}{2}$, and another vertex $t \in \V{v}{2}$ such that $s \ne t$.
Then, $\D{s}{t}{v} = \bar{H}(\{\D{s}{n}{v} : n \in \V{t}{1}\})$ if $\D{s}{n}{v} > 0$ for every 1-hop neighbor $n$ of vertex $t$, where $\bar{H}(\{\D{s}{n}{v} : n \in \V{t}{1}\})$ denotes the {\em harmonic mean} computed over $\{\D{s}{n}{v} : n \in \V{t}{1}\}$.
\begin{proof}
We prove this theorem considering the following two cases.
% in two cases, $s \in \LV{v}{1}$ or $s \in \LV{v}{2}-\LV{v}{1}$, separately:
\noindent [Case I. $s$ is a 1-hop neighbor of $v$ (i.e., $s \in \V{v}{1}$)]
% Let us assume there are $m$ 1-hop neighbors of $t$ in $\XN{v}$ and let $n_i$ denote the $i$th 1-hop neighbor of $t$ such that $\D{s}{n_i}{v} > 0$. In $\XN{v}$, each $n_i$ is also an 1-hop neighbor of $v$ (i.e., $n_i \in \LV{v}{1}$) and $\D{s}{n_i}{v} = {1}/{|\LV{s}{1} \cap \LV{n_i}{1}|}$ by Theorem \ref{theorem2}. 
% Furthermore, (i) $|\LV{s}{1} \cap \LV{n_i}{1}|=1/\D{s}{n_i}{v}$ represents the number of shortest paths from $s$ to $n_i$ (i.e., $\sigma_{sn_i}$) and the total number of shortest paths from $s$ to $t$ can be expressed as $\sigma_{st} = \sum_{i=1}^m 1/\D{s}{n_i}{v}$ since each $n_i$ shares an edge with $t$. On the other hand, (ii) the path $s-v-n_i-t$ is the only shortest path from $s$ to $t$ via both $v$ and $n_i$ since the length of the path is 3 and any other path from $s$ to $t$ via both $v$ and $n_i$ must be longer. It means that there are $m$ shortest paths from $s$ to $t$ via $v$ (i.e., $\sigma_{st}(v)=m$). By (i) and (ii), $\D{s}{t}{v} = \frac{m}{\sum_{i=1}^m 1/\D{s}{n_i}{v}} = \bar{H}(\D{s}{n}{v})$.
Let $n_i$ ($i=1,2,\cdots, m$) denote the $i$th 1-hop neighbor of $t$.
Then, each $n_i$ is a 1-hop neighbor of $v$ since, in $\XN{v}$, any 2-hop neighbor (including $t$) of $v$ can be connected only to a 1-hop neighbor of $v$ (Definition~\ref{def:xego-network}).
Thus, by Theorem \ref{theorem2}, $\D{s}{n_i}{v} = {1}/{|\V{s}{1} \cap \V{n_i}{1}|}$. 
Since $|\V{s}{1} \cap \V{n_i}{1}|$ represents the number of shortest paths from $s$ to $n_i$ (i.e., $\sigma_{sn_i}$) and each $n_i$ has an edge to $t$, the total number of shortest paths from $s$ to $t$ can be expressed as (i) $\sigma_{st} = \sum_{i=1}^m |\V{s}{1} \cap \V{n_i}{1}| = \sum_{i=1}^m 1/\D{s}{n_i}{v}$. 
On the other hand, the path $s-v-n_i-t$ is the only shortest path from $s$ to $t$ via both $v$ and $n_i$ since the length of the path is 3 and any other path from $s$ to $t$ via both $v$ and $n_i$ must be longer.
For this reason, (ii) there are $m$ shortest paths from $s$ to $t$ via $v$ (i.e., $\sigma_{st}(v)=m$).
By (i) and (ii), $\D{s}{t}{v} = \frac{m}{\sum_{i=1}^m 1/\D{s}{n_i}{v}} = \bar{H}(\D{s}{n}{v})$.
\noindent [Case II. $s$ is a 2-hop neighbor of $v$ (i.e., $s \in \V{v}{2}$)]
Let $n_i~(i=1,2,\cdots,m)$ denote the $i$th 1-hop neighbor of $t$. 
Let also $p_j~(j=1,2,\cdots,k)$ denote the $j$th 1-hop neighbor of $s$.
In $\XN{v}$, 2-hop neighbors (including $s$ and $t$) of $v$ can be connected only to a 1-hop neighbor of $v$.
For this reason, any shortest path from $s$ to $t$ must contain a shortest path from $p_j$ to $n_i$ for some $j$ and $i$ 
% Since $n_i$ and $p_j$ respectively shares an edge with $t$ and $s$, the shortest path from $n_i$ to $n_j$ for a pair of $i$ and $j$ always corresponds to the one from $s$ to $t$ 
(i.e., $\sigma_{st} = \sum_{i=1}^m \sum_{j=1}^k |\V{n_i}{1} \cap \V{p_j}{1}|$). 
For a pair of $i$ and $j$, on the other hand, the path $n_i-v-p_j$ is the only shortest path from $n_i$ to $p_j$ via $v$ and so is the path $t-n_i-v-p_j-s$ (i.e., $\sigma_{st}(v)=mk$). 
Therefore, 
\begin{eqnarray}
\D{s}{t}{v} 
&=& \frac{mk}{\sum_{i=1}^m \sum_{j=1}^k |\V{n_i}{1} \cap \V{p_j}{1}|} \nonumber \\ 
&=& \frac{mk}{\sum_{i=1}^m \sum_{j=1}^k 1/\D{n_i}{p_j}{v}} \nonumber \\ 
&=& \frac{m}{\sum_{i=1}^m \frac {\sum_{j=1}^k 1/\D{n_i}{p_j}{v}}{k}} \nonumber \\ 
&=& \frac{m}{\sum_{i=1}^m 1/\D{n_i}{s}{v}} \nonumber \\ 
&=& \frac{m}{\sum_{i=1}^m 1/\D{s}{n_i}{v}} \nonumber  \\
&=& \bar{H}(\D{s}{n}{v}).   \nonumber
\end{eqnarray}
\label{harmonic_eq}
\end{proof}
\end{theorem}
% \label{harmonic} 
% For a vertex $v$, its x-ego network $\XN{v}$, a vertex $s \in \V{v}{1} \cup \V{v}{2}$ and another vertex $t \in \V{v}{2}$, let $n_1, n_2, \cdots, n_i$ ($n_i \leq u$) denote the 1-hop neighbor vertices of vertex $t$. 
% Then, 
% \begin{equation}
% \D{s}{t}{v} =
%   \begin{cases}
%    0      &\text{if $\exists k, \D{s}{n_k}{v} = 0$} \\
%    \bar H (\D{s}{n_k}{v}) &\text{otherwise.}
%    \end{cases}
% \end{equation} where $\bar H_{st}$ is the harmonic mean for all the non-zero $f_{s{n_k}}$ values.
% \end{theorem}
% \begin{proof}
% ...\\
% ...\\
% ...\\
% ... \qed
% \end{proof}
\begin{example}
In Fig.~\ref{eego2}(b), % consider the dependency $\D{a}{h}{v}$ of $a \in \LV{v}{2}$ and $h \in \V{v}{2}$ on $v$. In this example, 
$\V{h}{1} = \{c,d\}$.
By Theorem \ref{theorem2}, $\D{a}{c}{v}=\frac{1}{2}$ and $\D{a}{d}{v}=1$.
Therefore, by Theorem \ref{theorem4}, $\D{a}{h}{v}\!=\!\bar{H}(\{\D{a}{n}{v} : n \in \V{h}{1}\})\!=\!\bar{H}(\{\D{a}{c}{v},\D{a}{d}{v}\})\!=\!\frac{2}{\frac{1}{\frac{1}{2}} + \frac{1}{1}}=\frac{2}{3}$.  
% For each 1-hop neighbor $n$ of $h$ (i.e., $n \in \V{h}{1} = \{ c,d\}$,  and from Theorem \ref{theorem2} it also knows $\D{a}{c}{v}=1/2$ and $\D{a}{d}{v}=1/1$ for $h$'s two 1-hop neighbors, $c$ and $d$. Therefore, we can get $\D{a}{h}{v}$=$\bar{H}(\D{a}{c}{v},\D{a}{d}{v})$=$2/3$. 
% Similarly, $\D{c}{e}{v}=\bar{H}(\D{c}{a}{v})=1/2$ where $c \in \LV{v}{1} (\subset \LV{v}{2})$ and $e \in \LV{v}{2}-\LV{v}{1}$, since $\LV{e}{1}=\{e,a\}$, $\D{c}{a}{v}=1/2$ and $\bar{H}(\D{c}{a}{v})=\D{c}{a}{v}$.
\end{example}
% These four properties will be used in the algorithm proposed in Section \ref{computation} to compute the x-ego betweenness of a DTN node.